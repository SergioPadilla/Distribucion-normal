\documentclass[paper=a4, fontsize=11pt, spanish]{scrartcl}
\usepackage{fourier}
\usepackage{hyperref}
\usepackage{dsfont}
\usepackage{amssymb}

\usepackage{sectsty}
\allsectionsfont{\normalfont\scshape}
\setlength\parindent{0pt}

\usepackage{fancyhdr}
\pagestyle{fancyplain}
\fancyhead{}
\fancyfoot[L]{}
\fancyfoot[C]{}
\fancyfoot[R]{\thepage}
\renewcommand{\headrulewidth}{0pt}
\renewcommand{\footrulewidth}{0pt}
\setlength{\headheight}{13.6pt}

\usepackage[spanish,es-noquoting,es-lcroman]{babel}
\usepackage[utf8]{inputenc}
\usepackage[T1]{fontenc}
\selectlanguage{spanish}

%----------------------------------------------------------------------------------------
%	TÍTULO
%----------------------------------------------------------------------------------------
% Título con las líneas horizontales, nombres y fecha.

\newcommand{\horrule}[1]{\rule{\linewidth}{#1}}

\title{
  \normalfont \normalsize
  \textsc{Universidad de Granada.} \\ [25pt]
  \horrule{0.5pt} \\[0.4cm]
  \huge Distribución Normal \\
  \horrule{2pt} \\[0.5cm]
}

\author{Sergio Padilla López \and Iván Calle Gil\\}

\date{\normalsize\today}


%----------------------------------------------------------------------------------------
%	DOCUMENTO
%----------------------------------------------------------------------------------------

\begin{document}
\maketitle
\tableofcontents

\newpage

\section{Función de densidad}
La distribución normal, también llamada distribución de Gauss es una distribución de probabilidad cuya función de densidad es la siguiente:

$$ f(x|\mu,\sigma^2)=\frac{1}{\sigma\sqrt{2\pi}}e^{-\frac{(x-\mu)^2}{2\sigma^2}}   $$

donde, $x \in \mathds{R}$, $\mu$ es la media y $\sigma^2$ es la varianza de la variable aleatoria.

Por tanto, notamos a la distribución normal como, $n(\mu,\sigma^2)$.

\section{Función de distribución}
Una vez conocida la función de densidad, vamos a escribir la función de distribución como sigue:

$$
P(X \leq x) = \int_{-\infty}^{x} \frac{1}{\sigma\sqrt{2\pi}}e^{-\frac{(t-\mu)^2}{2\sigma^2}} \cdot dt
$$

Veamos que si una variable aleatoria $X \backsim n(\mu,\sigma^2)$ la variable aleatoria $Z=\frac{x-\mu}{\sigma}$ es $n(0,1)$.

$$
P(Z \leq z) = P(\frac{x-\mu}{\sigma} \leq z) = P(X \leq z\sigma + \mu) = \frac{1}{\sigma\sqrt{2\pi}} \int_{-\infty}^{z\sigma + \mu} e^{-\frac{(x-\mu)^2}{2\sigma^2}} \cdot dx
$$
haciendo $t=\frac{x-\mu}{\sigma}$:

$$
\frac{1}{\sigma\sqrt{2\pi}} \int_{-\infty}^{z\sigma + \mu} e^{-\frac{(x-\mu)^2}{2\sigma^2}} \cdot dx = \frac{1}{\sqrt{2\pi}} \int_{-\infty}^{z} e^{-\frac{t^2}{2}} \cdot dt
$$

y por tanto, $P(Z\leq z)$ sigue n(0,1).

\section{Función Generatriz de momentos}
Definimos la función generatriz de momentos como la esperanza de la variable $e^{tX}$:

$$ M_x(t) = E[e^{tX}] = \frac{1}{\sigma \sqrt{2\pi}} \int_{-\infty}^{\infty} e^{-\frac{(x-\mu)^2}{2\sigma^2}}e^{tX} \cdot dx $$

$$
= \frac{1}{\sigma \sqrt{2\pi}} \int_{-\infty}^{\infty} e^{\frac{2\sigma^2tX-(x^2+\mu^2-2x\mu)}{2\sigma^2}} \cdot dx
$$

$$
= \frac{1}{\sigma \sqrt{2\pi}} \int_{-\infty}^{\infty} e^{-\frac{x^2-2(\sigma^2t+\mu)x + \mu^2}{2\sigma^2}} \cdot dx
$$

$$
= \frac{1}{\sigma \sqrt{2\pi}} \int_{-\infty}^{\infty} e^{-\frac{x^2-2(\sigma^2t+\mu)x + \mu^2 - (\sigma^2t+\mu)^2+(\sigma^2t+\mu)^2}{2\sigma^2}} \cdot dx
$$

$$
= \frac{1}{\sigma \sqrt{2\pi}} \int_{-\infty}^{\infty} e^{-\frac{(x-(\sigma^2t+\mu))^2 + \mu^2 - (\sigma^2t+\mu)^2}{2\sigma^2}} \cdot dx
$$

$$
= \frac{1}{\sigma \sqrt{2\pi}} \int_{-\infty}^{\infty} e^{-\frac{(x-(\sigma^2t+\mu))^2 - \sigma^2(\sigma^2t^2+2\mu t)}{2\sigma^2}} \cdot dx
$$

$$
= \frac{1}{\sigma \sqrt{2\pi}}e^\frac{\sigma^2t^2+2\mu t}{2} \int_{-\infty}^{\infty} e^{-\frac{(x-(\sigma^2t+\mu))^2}{2\sigma^2}} \cdot dx
$$

haciendo el cambio de variable: $v=\frac{x-(\sigma^2t+\mu)}{\sigma}$ ; $dv=\frac{1}{\sigma} dx$:

$$
= \frac{1}{\sigma \sqrt{2\pi}}e^\frac{\sigma^2t^2+2\mu t}{2} \int_{-\infty}^{\infty} e^{-\frac{v^2}{2}}\sigma \cdot dv
$$

$$
= \frac{1}{\sqrt{2\pi}}e^\frac{\sigma^2t^2+2\mu t}{2} \int_{-\infty}^{\infty} e^{-\frac{v^2}{2}} \cdot dv
$$

y sabiendo que: 

$$
\frac{1}{\sqrt{2\pi}} \int_{-\infty}^{\infty} e^{-\frac{v^2}{2}} \cdot dv = 1
$$

tenemos que:

$$
M_x(t) = e^{\frac{\sigma^2t^2}{2}+\mu t}
$$

\section{Esperanza}
La esperanza matemática de la distribución normal se corresponde con el momento de orden 1 que se calcula como:

$$
E[X]=E[X^2]-E[X]^2=M_x'(0)
$$

Calculemos la primera derivada de la función generatriz de momentos:

$$
M_x'(t)=(\sigma^2t+\mu)e^{\frac{\sigma^2t^2}{2}+\mu t}
$$

Por tanto:

$$
E[X]=\mu
$$

\section{Varianza}
La varianza se corresponde con el momento de orden 2:

$$
\sigma^2=E[X^2]-E[X]^2=M_x''(0)
$$

Calculemos la segunda derivada de la función generatriz de momentos:

$$
M_x''(t)=((\sigma^2t+\mu)^2+\sigma^2)e^{\frac{\sigma^2t^2}{2}+\mu t}
$$

Por tanto, tenemos:

$$
\sigma^2=\mu^2+\sigma^2
$$

\end{document}
